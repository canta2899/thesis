\label{ch:background}

In order to define a common ground for the reader, this chapter introduces the main contents on which this work is based. For this purpose, a brief description about the current \textbf{state of knowledge} in terms of fall detection and the role of the \textbf{Electrodermal Activity} is proposed. Additionally, the concept of \textbf{Wearable Device} is briefly introduced.

% TODO evaluate the title, might need to be changed
\section{Benefits and Progress of Fall Detection Systems }\label{sec:fallintro}

% file:///Users/andrea/Downloads/sensors-21-05134-v2.pdf
In the field of research, a fall is described as an \textbf{unpredicted event} leading a subject to rest on the floor level \cite{Lamb1}. On that basis, the activity of Fall Detection has referred to the identification of a fall through sensors and cameras in order to request assistance.

Over the past twenty years, researchers have worked on data retrieved from individual sources (such as gyroscopes and accelerometers) at the beginning and subsequently started investigating \textbf{fusion based approaches}. Those involve the collection of multiple data of various nature and lead to more robust and increasingly accurate classification models. 

Vision based approaches gained importance over the last decade, thanks to the improvements in the world of \textbf{Machine vision} and to new technologies such as \textbf{depth cameras} \cite{elderlySurvey}. Major advancements in fall detection systems have, in fact, been reached by leveraging both vision and sensor-fusion based solutions. 

The latter gives rise to various difficulties in terms of data retrieval, synchronization, storage, processing and analysis, which may be overcome thanks to the continuous advancements in technology.

% Chen classification of fall, maybe to much for introduction
% Furthermore, various classifications of falls have been outlined. An accurate set of categories was described by Chan \textit{et al., 2018}~\cite{Chen1}, whose study about the specifications of built-in accelerometers in smartphones on fall detection performance divides falls in the following categories: 

% \begin{itemize}
%     \item Fall lateral left and lie on the floor
%     \item Fall lateral left and sit up from the floor
%     \item Fall lateral right and lie on the floor
%     \item Fall lateral right and sit up from the floor
%     \item Fall forward and lie on the floor
%     \item Fall backward and lie on the floor
% \end{itemize}

Researchers had also pointed out other important aspects that should be considered, such as a complete and detailed classifications of various kind of falls (in order to acquire different sets of information), their duration, the age and gender of the individual, together with its \textbf{mental} and \textbf{physical} condition.
Because most of the research conducted at present does not consider all these aspects together, the models obtained tend not to generalize well in real-life situations, leading to \textbf{inaccuracy}.

On this purpose, data retrieved from biosensors may be evaluated in order to provide additional information regarding an individual's condition in the context of a fall.

\section{Skin Conductance}\label{sec:edaintro}

With the objective of providing a criteria to detect falls with higher degrees of accuracy, an analysis turned towards the description of the emotional condition of the individual may provide useful information. It is, in fact, reasonable to assume that, after an unexpected fall, a person might experience a condition of \textbf{stress} and \textbf{fear}.

On that basis, an approach for the description of the previously mentioned condition may rely on the analysis of the \textbf{Autonomic Nervous System} activity. This constitutes an essential component of the peripheral nervous system aimed at controlling smooth muscles and glands functionality in the human body \cite{ansWiki}.

The three branches that compose the autonomic nervous system are:

\begin{itemize}
    \item The \textbf{Sympathetic nervous system} (responsible for stimulating the \textit{fight or flight} response, which prepares the organism to address situations of danger or emotional distress)
    \item The \textbf{Parasympathetic nervous system} (responsible for regulating unconscious actions of the organism)
    \item The \textbf{Enteric nervous system} (responsible for the nervous activity in the digestive system)
\end{itemize}

The activity of sweat glands endures a variation directly proportional to the liveliness of the sympathetic nervous system and its analysis defines a valid criteria to depict its response to external stimuli.

At present, a promising alternative for the non-invasive assessment of sympathetic control of the autonomic nervous system consists in the analysis of the \textbf{Electrodermal Activity} \cite{edaIntro1}, whose characteristics are discussed in the following chapters.

\section{Wearable Devices}\label{sec:wearables}

% https://medcitynews.com/2021/07/how-wearable-devices-empower-healthcare-providers/

As part of fall detection studies, activities of data collection and retrieval are usually performed by various equipment normally referred as \textbf{Wearable Devices}. These are electronic devices worn by individuals to track various types of information, especially \textbf{biometric} \cite{wearablesDefinition}.

Wearables constitute a great innovation in the world of healthcare and research, providing a sound strategy to ubiquitously collect physiological data, even from uncontrolled and real-life scenarios.

Some of the most well-known consumer grade wearables are devices such as the Apple Watch and Fitbit Bands, whose firmware already provides different ways to gather health data that can be later shared with doctors and healthcare personnel \cite{wearablesBest}. 

