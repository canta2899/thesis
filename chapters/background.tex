\label{ch:background}

In order to define a common ground for the reader, this chapter introduces the main contents on which this work is based. For this purpose, a brief introduction about the current \textbf{state of knowledge} in terms of fall detection and the role of the \textbf{Electrodermal Activity} is proposed. Additionally, the concept of \textbf{Wearable Device} is briefly introduced.

% TODO evaluate the title, might need to be changed
\section{Benefits and Progress of Fall Detection Systems }\label{sec:sectionname}

% file:///Users/andrea/Downloads/sensors-21-05134-v2.pdf
In the field of research, a fall is described as an \textbf{unpredicted event} leading a subject to rest on the floor level~\cite{Lamb1}. On that basis, the activity of Fall Detection has referred over the years to the identification of a fall through cameras and sensors in order to request assistance.

Furthermore, various classifications of falls have been outlined. An accurate set of categories was described by Chan \textit{et al., 2018}~\cite{Chen1}, whose study about the specifications of built-in accelerometers in smartphones on fall detection performance divides falls in the following categories: 

\begin{itemize}
    \item Fall lateral left and lie on the floor
    \item Fall lateral left and sit up from the floor
    \item Fall lateral right and lie on the floor
    \item Fall lateral right and sit up from the floor
    \item Fall forward and lie on the floor
    \item Fall backward and lie on the floor
\end{itemize}

Researchers had also pointed out other important aspects that should be considered while evaluating falls, such as the duration of the fall and the age, gender, mental condition and physical condition of the individual.


