\label{ch:analysis}

With the purpose of providing a deeper understanding of how some of the most relevant datasets regarding fall detection were collected, an analysis was performed in order to identify common \textbf{patterns}, the \textbf{technologies} employed and the \textbf{results obtained}. Afterwards, the previously mentioned \textbf{Electrodermal Activity} and its potential implications in the context of fall detection was investigated.

\section{Multimodal datasets for fall detection systems}\label{sec:datasets}

Despite the fact that a wide range of falls datasets have been made available throughout the years, four or them were selected for the purpose of this analysis as a consequence of their relevance in the research environment.

\subsection{UMAFall - A Multisensor Dataset for the Research on Automatic Fall Detection}\label{subsec:umafall}

The \textbf{UMAFall} dataset gained considerable interest since its publication (happened in 2017). The primary difference from other datasets was, in fact, related to the way Casilari \textit{et al., 2018}~\cite{umafall} approached the data collection stage, which involved the usage of multiple units of the same sensor.

