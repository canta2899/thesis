\label{ch:conclusions}

This dissertation aimed to provide an assessment related to the feasibility of the Electrodermal Activity data analysis in the context of fall detection. For this purpose, the concepts of \textbf{Skin Conductance} and \textbf{Wearable Devices}, as well as the most important aspects of fall detection systems, were briefly introduced and later examined. More specifically, the common technologies employed in the context of fall detection were described and some of the most relevant results in the field of research were, then, analyzed. Afterwards, a description regarding the nature of the \textbf{Electrodermal Activity signals} and their characteristics was provided, together with a careful analysis related to the implications of Skin Conductance in research contexts.

The approach adopted for the implementation of a Wearable Device aimed at providing high resolution EDA data logging was, then, described. For this purpose, after an analysis related to the features of the \textbf{BITalino EDA Sensor}, the hardware and software choices made in order to provide a simple and practical implementation were outlined. The device was tested and validated in an apposite experimental context. Besides issues related to its frailty were identified, a first data retrieval session was possible. A subsequent analysis of the collected EDA data was proposed and the results obtained suggested the need of further investigations in order to assess possible correlations between the Electrodermal Activity and fall detection.

Thanks to the collaboration with \textbf{Movisens GmbH}, a high-end wearable device was rented in order to collect additional data in a second experimental session. Several falls were performed and the retrieved data was later processed and analyzed.

Besides typical EDA features have been detected during the overall measurement, the results obtained have not demonstrated a clear correlation with the falling activities performed by the participant involved in the experimentation. Moreover, the acquired signal was strongly influenced by motion artifacts which did not allow a clean data collection. For this purpose, better processing technologies (such as specific filters) and acquisition criteria should be provided in order to allow the assessment of the Electrodermal Activity even in contexts of high mobility.

Finally, the observation of specific data acquired during the falls examined suggested that an extensive analysis related to the Electrodermal Activity measurement during \textbf{unexpected falls} should be assessed. In fact, self initiated and simulated falls may be cause of psychophysiological reactions that differ from the ones observable in real-life scenarios. However, obtaining these kind of data at present results considerably difficult because of \textbf{economical}, \textbf{practical} and \textbf{ethical reasons}.

Hopefully, this work will help researchers with the intent of providing further acquisitions of the Electrodermal Activity data in order to configure experimentations aimed at capturing bio-signals similar to the ones that would be retrieved in uncontrolled environments.


