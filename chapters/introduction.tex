\label{ch:introduction}

Falling may be experienced as one of the most damaging events, especially for elderly people. According to the World Health Organization Newsroom \cite{WhoData}, an estimated 684.000 individuals die every year from falls globally, while 37.3 million falls turn out to be severe enough to require medical attention. 
Over the years, an urgent need for fall detection system has led researchers to the development of increasingly accurate solutions involving the usage of different sensors and various techniques for data processing, collection and analysis.
Furthermore, thanks to the rapid development of the Internet of Things, the employment of \textbf{sensors networks} allowing the fusion of heterogeneous data has been regarded as an effective method to address the problem of fall detection.

Whereas data retrieved from devices such as \textbf{Inertial Measurement Units} constitutes the primary area of study, over the past few years researchers have developed a growing interest towards the analysis of bio-signals in order to depict the consequences that a fall may cause to the human body both at the \textit{physical} and \textit{mental} level.

This dissertation describes the most advanced and accurate results obtained by the research community and analyzes the feasability of the analysis of the \textbf{Electrodermal Activity} to detect the variation of stress condition during a fall. 

% TODO cite https://www.who.int/news-room/fact-sheets/detail/falls
% TODO furthermore or additionally?
% TODO describe chapters
