\label{ch:introduction}

Falling may be experienced as one of the most damaging events, especially for elderly people. According to the World Health Organization Newsroom, an estimated 684.000 individuals die every year from falls globally, while 37.3 million falls turn out to be severe enough to require medical attention \cite{WhoData}. 
Over the years, an urgent need for fall detection system has led researchers to the development of increasingly accurate solutions involving the usage of different sensors and various techniques for data processing, collection and analysis.
Furthermore, thanks to the rapid development of the Internet of Things, the employment of \textbf{sensor networks} allowing the fusion of heterogeneous data has been regarded as an effective method to address the problem of fall detection.

Whereas data retrieved from devices such as \textbf{Inertial Measurement Units} constitutes the primary area of study, over the past few years researchers have developed a growing interest towards the analysis of bio-signals in order to depict the consequences that a fall may cause to the human body at both \textit{physical} and \textit{mental} level.

This dissertation analyses the feasibility of the \textbf{Electrodermal Activity} Data Analysis in order to detect variation of stress conditions in the context of fall detection. 

In \textbf{Chapter 2}, background information is provided in order to contextualize the concepts of Fall Detection Systems, Electrodermal Activity and Wearable Devices. Modern approaches employed in the field of fall detection are, then, described and analysed in \textbf{Chapter 3}. Afterwards, an in-depth study related to the Electrodermal Activity and its use cases is outlined in \textbf{Chapter 4}. The development of an open-source wearable device for high resolution EDA data logging and the results obtained from a first trial session are, then, detailed in \textbf{Chapter 5}. Thereafter, further investigations related to the usage of EDA data for fall detection purposes are provided in \textbf{Chapter 5}, which describes the results obtained in a second trial session that involved the usage of a well-known commercial device for data retrieval. Finally, conclusions in the merits of the obtained results are presented in \textbf{Chapter 6}.
