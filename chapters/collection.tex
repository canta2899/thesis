\label{ch:collection}

%               ROADMAP
% =====================================
%
% we decided to use a commercial product
% movisens and the edamove4
%     - movisens
%     - edamove4
%     - sensor usage and data retrieval
%     - accessory tools
%     - unisens data format
%     - data analyzer and implemented features
% data Collection (falls performed and the hospital context)
% results
%

In order to further investigate the result obtained during the first experimentation session, a second testing trial was performed. This allowed the collection of additional data in order to draw conclusions in relation to the \textbf{feasibility} of the Electrodermal Activity analysis for classification purposes in the context of fall detection.

In this case, a commercial and publicly acknowledged device was employed in order to overcome the issues concerning the sensitivity to shocks and impacts of the BITalino device and, in addition, to acquire further bio-signals along with the Electrodermal Activity data.

\section{Movisens GmbH and the EdaMove Sensor}\label{sec:movisens}


