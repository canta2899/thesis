\label{ch:collection}

In order to further investigate the results obtained during the first experimentation session, a second testing trial was performed. This allowed the collection of additional data in order to draw conclusions in relation to the \textbf{feasibility} of the Electrodermal Activity analysis for classification purposes in the context of fall detection.

In this case, a commercial and publicly acknowledged device was employed in order to overcome the issues concerning the sensitivity to shocks and impacts of the BITalino device and, in addition, to acquire further bio-signals along with the Electrodermal Activity data.

\section{Choosing a High End Electrodermal Activity Sensor}\label{sec:movisens}

The device employed consists of the \textbf{EdaMove 4} sensor by \textbf{Movisens GmbH}, which is one of the most accurate and popular wearable devices aimed at the Electrodermal Activity measurement together with others such as the previously mentioned \textbf{Empatica E4}. In the following sections, a brief overview of the product and its features in proposed.

\subsection{Movisens GmbH}\label{subsec:movisens-company}

Movisens GmbH is a German company deemed as a global leader in ambulatory assessment solutions \cite{movisens}, which offers several services such as workshops, consults, customized products and software in order to support researchers and provide technologies for the healthcare sector.
The current catalog offers several instruments for data collection, integration, processing and analysis. More specifically, a list of the currently available products is reported in table \ref{toc:movisens-products}.

\begin{table}[H]
\centering
\begin{tabular}{ll}
    \hline
    Name                     &  Description \\
    \hline
    \textbf{Move 4}          & A 10-Axis IMU that integrates a temperature sensor \\
    \textbf{EcgMove 4}       & A Move 4 unit that integrates ECG data \\
    \textbf{EdaMove 4}       & A Move 4 unit that integrates EDA data \\
    \textbf{LightMove 4}     & A Move 4 unit that integrates the acquisition of ambient light measurement \\
    \textbf{SensorTrigger}   & A mobile solution for activity-triggered data logging \\
    \textbf{movisensXS}      & A mobile solution for Experience Sampling purposes \\
    \textbf{DataMerger}      & A desktop-based application for the integration of heterogeneous data \\
    \textbf{DataAnalyzer}    & A desktop-based application to analyze and report the acquired data \\
    \hline
\end{tabular}
\caption{List of the products implemented by Movisens GmbH}
\label{toc:movisens-products}
\end{table}

Additionally, the company provides several solutions that allow researchers and students to utilize the needed instrumentation by requiring a free, limited rent for a specific product. In this case, the company has consented a rental of the \textbf{EdaMove 4} sensor for the required period of time. 

\subsection{The EdaMove 4 Sensor}\label{subsec:edamove4}

Across all the products developed by Movisens, the EdaMove 4 is the one that implements measurement functionalities for the Electrodermal Activity. It consists of a 62,3 x 38,6 x 11,5 mm case with a 26 g weight \cite{edamove4} that provides a Micro-USB connection (in order to connect the unit to the \textit{Sensor Manager} software) and two snap-button connectors in order to attach the positive and negative electrodes. Moreover, a proper wrist band with pre-connected electrodes and specific mount points for the EdaMove 4 sensor allows the device to be worn seamlessly on the wrist or the ankle.


\begin{figure}[htp]
    \centering
    \subfloat[Electrodes Placement]{%
        \includegraphics[width=0.4\textwidth]{./images/edamove1.jpeg}%
    }%
    \hfill%
    \subfloat[The Sensor Unit Employed]{%
        \includegraphics[width=0.4\textwidth]{./images/edamove2.jpeg}%
    }%
    \caption{The EdaMove 4 Sensor}
    \label{fig:edamovesensor}
\end{figure}

Other than acquiring EDA measurement, the unit also offers:

\begin{itemize}
    \item A \textbf{3D Accelerometer}
    \item A \textbf{Rotation Rate} sensor
    \item A \textbf{Pressure} sensor
    \item A \textbf{Temperature} sensor
\end{itemize}

More specifically, the unit implements an exosomatic acquisition technique by applying a standard voltage of 0.5 V DC. The internal Analog to Digital Converter provides a 14 bit resolution and the output rate corresponds to 32 Hz for a 8 Hz bandwidth.

Furthermore, the 3D accelerometer provides a $\pm$ 16 g range with an output rate of 64 Hz, while the rotation rate sensor offers a $\pm$ 2000 dps range, with a 64 Hz output rate and a 70 mdps resolution. The pressure sensor provides, instead, a measurement range between 300 and 1100 hPa, with a 8 Hz output rate and a 0.03 hPa resolution. Finally, the temperature sensor provides an output rate of 1 Hz.

The EdaMove 4 mounts a capacious Lithium battery which provides a constant 3.7 voltage and offers a continuous runtime of almost four days. An internal memory of 4 Gigabytes offers, instead, a maximum recording capacity of 4 weeks, allowing the usage of the sensor in varied contexts, from controlled experimentations to uncontrolled, domestic environments. 

Finally, the EdaMove 4 has been employed in several scientific publications that overall validated its functionalities and accuracy.

\subsection{Sensor Usage}\label{subsec:edamove4-usage}

In order to organise a measurement session and subsequently extract the obtained data, Movisens provides a specific software toolchain, which is cross-compatible between all the Move sensors. 

More specifically, the sensors can be configured through the \textbf{Sensor Manager} software, which allows the definition of personal data in relation to the individual and the specification of parameters such as the \textbf{duration} and the \textbf{start time} of the following measurement. Furthermore, the software detects whether the unit has non-retrieved measurements in its storage and provides functionalities to transfer the acquired dataset to the machine of the user. The data acquisition criteria can be configured in order to match one of the two modalities proposed: 

\begin{itemize}
    \item \textbf{Plain CSV} files
    \item \textbf{Unisens} File Format 
\end{itemize}

Besides Plain CSV is included as an option, the company highly suggests the usage of the Unisens format in order to exploit the functionalities that the standard implements. The latter is, in fact, compatible with both the \textbf{Unisens Viewer} and the \textbf{Data Analyzer} software. The first consists of a free and convenient tool that allows to plot and pre-process the raw data acquired (providing functionalities to remove unwanted sections and set specific markers in order to identify events of interest). The second one, instead, consists of a paid software that implements several algorithms and data processing techniques for the analysis of all the data acquired by Movisens products.

\subsection{The Unisens Data Format}\label{subsec:unisens}

One of the main difficulties that are required to be addressed when collecting data from multiple and heterogeneous sources is related to the storage and representation of the acquired information. More specifically, data can be retrieved at different sample rates or at specific time intervals and requires to be later synchronized e recomposed in order to be represented, analyzed and, eventually, classified. This problematic has been addressed in a multitude of different ways by researches. For example, as stated in \ref{subsubsec:upfall-technologies}, in the case of the UP-Fall dataset, a unique sample rate was defined in order to allow data synchronization from different sources. This strategy, however, required to lower the sample rate of certain devices which could have provided much more accuracy by configuring higher sample rates.

With these aims, the \textbf{FZI Research Center for Information Technology} and the \textbf{Institute for Information Processing Technology} at the University of Karlsruhe developed \textbf{Unisens}, a universal format standard that provides a shared criteria to collect multi sensor data \cite{unisens}. The latter is licensed under the LGPL license and provides the following abilities and features: 

\begin{itemize}
    \item Handling of continuous signals, as well as annotations and values
    \item Separate storage of multiple data channels
    \item Ability to synchronize data collected at various sample rates
    \item A layer of separation between raw data and meta information
    \item A flexible criteria to organise data in logical groups
    \item A human and machine readable meta information source file
    \item Ease of use, even in the case of embedded systems
\end{itemize}

The structure of a Unisens dataset involves the presence of several independent data sources stored as CSV or binary files which are, then, \textbf{aggregated} through a single XML file. The latter provides generic meta information in relation to additional sensor-related data, the initial time-stamp of the measurement, the source file of each data channel (with the required parameters in order to allow a correct parsing) and more.

\vspace{5mm}

\begin{figure}[h!]
    \centering
    \includegraphics[width=15cm]{./images/unisens_viewer.png}
    \caption{The Unisens Viewer Application}
    \label{fig:unisens-viewer-gui}
\end{figure}

As stated previously, the Movisens software toolchain provides a complete integration of the Unisens data format, which was therefore used in order to acquire and store fall detection related data. Additionally, Unisens provides support for common general-purpose programming languages such as \textbf{Python}, \textbf{Java}, \textbf{C\#} and \textbf{Matlab} other than offering the previously mentioned Unisens Viewer software, a desktop based application compatible with the Windows Operative System that allows users to interact with datasets through a graphical user interface. An extract of the content of a Unisens XML file has been reported in figure \ref{fig:unisens-xml}.

\vspace{20mm}

\begin{figure}[H]
\begin{minted}{xml}

<?xml version="2.0" encoding="UTF-8" standalone="no"?>
<unisens comment="Description of the measurement" duration="1186.0"
         timestampStart="2021-12-02T15:12:01.569"
         ...>
    <customAttributes>
        <customAttribute key="sensorLocation" value="right_wrist"/>
        ...
        <customAttribute key="sensorTimeDrift" value="-3.7783"/>
    </customAttributes>
    <signalEntry adcResolution="16"
                 comment="acc"
                 contentClass="acc"
                 dataType="int16"
                 id="acc.bin"
                 lsbValue="0.00048828125"
                 sampleRate="64"
                 unit="g">
        <binFileFormat endianess="LITTLE"/>
        <channel name="accX"/>
        <channel name="accY"/>
        <channel name="accZ"/>
    </signalEntry>
    ...
</unisens>
\end{minted}
\caption{A brief extract of a Unisens XML file}
\label{fig:unisens-xml}
\end{figure}

\newpage

\section{Data Collection}\label{sec:edamove4-data-collection}

\subsection{Organisation of the Session}\label{subsec:session-org}

The data collection session was performed at the \textbf{Simulation Center} of the Santa Maria della Misericoria Hospital in Udine and involved the participation of a single 49 years old male and healthy individual. A single EDA Move 4 sensor was employed and worn on the wrist of the dominant arm of the participant, as depicted in figure \ref{fig:edamovesensor}. All the activities performed have been reported in table \ref{toc:falls-performed-edamove}. Moreover, like for the previous session, it is important to note that all the falls were self-initiated and performed on purpose in a completely safe environment. Finally, in some cases the same fall was performed multiple times.

\begin{table}[H]
\centering
\begin{tabular}{ll}
    \hline
    Name         & Description \\
    \hline
    Bed          & Falling while getting up from a bed \\
    BedSX        & Falling off the left side of the bed while getting up\\
    BedDX        & Falling off the right side of the bed while getting up \\
    SlumpingSX   & Slumping on the left side of the bed while falling  \\
    SlumpingDX   & Slumping on the right side of the bed while falling  \\
    SidewardsSX  & Standing and subsequently falling on the left side \\ 
    SidewardsDX  & Standing and subsequently falling on the right side \\ 
    Bending      & Falling while picking up an object from the ground \\ 
    Backwards    & Standing and subsequently falling backwards \\
    Front        & Standing and subsequently falling anteriorly  \\
    \hline
\end{tabular}
\caption{List of the falls performed}
\label{toc:falls-performed-edamove}
\end{table}

\subsection{Techniques Adopted}\label{subsec:session-techniques}

In this case, a \textbf{unique} and \textbf{continuous} measurement was acquired for the whole duration of the session. This strategy was adopted in order to provide the ability to analyze single event-related episodes as well as the whole evolution of the Electrodermal Activity throughout the entire duration of the experimentation. Moreover, timestamps were acquired in order to mark the beginning and the end of each fall and facilitate the subsequent process of separation of different events.

The acquired data (which also included information retrieved from the other sensors that compose the EdaMove 4) was, then, extracted and preprocessed in order to extract the raw EDA data (which was already converted in $\mu S$ by the sensor itself) and separate each fall. With these aims, a data processing algorithm was programmed and executed on a \textbf{Python Jupyter Notebook} by using the \textbf{PyUnisens} library.

Since timestamps have been acquired (by employing a stopwatch) in order to mark the beginning and the ending of each measurement, these were used in order to extract the data related to each fall. Since the EDA trace obtained from the Unisens dataset, as a measurement of fixed sample rate, did not provide time-related information, the \textbf{Measurement Start Timestamp} was used. This constitutes part of the standard meta information included in the Unisens XML file. More specifically, the EDA signal (which, in Python, is decoded as a Numpy array) was grouped in sub-arrays of fixed size, corresponding to the sample rate (which is, as stated in \ref{subsec:edamove4}, of 32 Hz). A time indication was, then, assigned (with an increment rate of one second) to every group. The latter procedure made possible the usage of the acquired time information in order to isolate each measurement.

\subsection{Results}\label{subsec:results}

The following sections report the most relevant results observed, which have been analyzed and plotted using the Python Programming Language and the previously mentioned \textbf{NeuroKit2} library. For this purposes, minor modifications were applied to the NeuroKit2 source code in order to fine tune specific parameters that provide cleaner results. All the plots were, then, produced by integrating the functionalities of the previously mentioned notebook. 

\subsubsection{Bed Fall}\label{subsubsec:bed-fall}

The first fall performed was the so-called \textbf{Bed Fall}. Three different variants of the latter were performed, which together reflect the most common bed-related fall that involve the majority of people, especially in the case of elderlies. These respectively concern: 

\begin{itemize}
    \item The act of falling while getting up from the bed, which may be caused by an unexpected inability to stand up 
    \item The act of falling due to the attempt to reach an object on the floor while lying on the bed, respectively on the left and right size
\end{itemize}

The results obtained have been reported in figures \ref{fig:movisens-bed}, \ref{fig:movisens-beddx} and \ref{fig:movisens-bedsx}. More specifically, with the same methodology adopted for Chapter \ref{ch:implementation}, the two signals components were separated through a deconvolution approach and plotted separately, together with the raw signal.

\begin{figure}[h!]
    \centering
    \includegraphics[width=\textwidth]{./images/movisens/Bed.png}
    \caption{EDA Trace for the Bed Fall}
    \label{fig:movisens-bed}
\end{figure}

The trace corresponding to the \textbf{Bed Fall} does not depict any increase in the levels of stress or arousal. Besides the raw signal shows a slight increase, it is clear from the separation of the two components that this is completely dictated by the influence of the phasic component. Because the Skin Conductance Level does not show any increase during the whole measurement duration, it is reasonable to assume that the activity described by the phasic component is mainly influenced by the hand movements and the impact of the sensor with the ground.

\begin{figure}[h!]
    \centering
    \includegraphics[width=\textwidth]{./images/movisens/BedDX.png}
    \caption{EDA Trace for the BedDX Fall}
    \label{fig:movisens-beddx}
\end{figure}

\begin{figure}[h!]
    \centering
    \includegraphics[width=\textwidth]{./images/movisens/BedSX.png}
    \caption{EDA Trace for the BedSX Fall}
    \label{fig:movisens-bedsx}
\end{figure}

The traces for the two other falls (respectively, the \textbf{BedDX} and \textbf{BedSX} one) have, instead, a similar pattern. More specifically, a \textbf{slight} and therefore \textbf{poorly significant} increase is observed for the tonic component, while a varying and eventful trace is described by the phasic component. Besides the raw signal of both falls shows an increase of almost 2 $\mu S$, it is possible to assume (by observing the tonic activity) that the cause may be mainly found in the hand movements produced by the individual. In both cases, in fact, the participant landed on the ground by actively putting all of his weight on the hands and therefore causing the electrodes to acquire more event-related stimuli.

In all the three cases examined, the Electrodermal Activity measurement did not, therefore, provide a clear and useful pattern for classification purposes.

\subsubsection{Slumping Fall}\label{subsubsec:slumping-fall}

After the bed falls, the slumping falls were performed. These involve the activity of slumping on the ground while failing the attempt of using the bed as a support in order to avoid keeling over. The action configures itself as a pattern that involves elderly and disabled people. Even in this case, the fall was performed for both the left and right side.

\begin{figure}[h!]
    \centering
    \includegraphics[width=\textwidth]{./images/movisens/SlumpingSX.png}
    \caption{EDA Trace for the SlumpingSX Fall}
    \label{fig:movisens-slumpsx}
\end{figure}

\begin{figure}[h!]
    \centering
    \includegraphics[width=\textwidth]{./images/movisens/SlumpingDX.png}
    \caption{EDA Trace for the SlumpingDX Fall}
    \label{fig:movisens-slumpdx}
\end{figure}

Both the measurements show a slight increase in the Skin Conductance Level, which was higher in the case of the right fall. At the same time, though, the right fall describes a more irregular activity of the phasic component. The latter may have been caused by motion artifacts happened during the falls and consequently have influenced the growth of the signal level. Both the traces do not, however, describe an EDA trace referable to a variation in the emotional condition, which would provide clear and isolated peaks with a significant increase of the SCL component.

\subsubsection{Bending Fall}\label{subsubsec:bending-fall}

The \textbf{Bending Fall} was, then, performed. The latter involves the activity of falling while trying to reach an object in order to grab it. It is, again, a common pattern for elderly people, but, may result significant even for young and healthy individuals.

\vspace{10mm}

\begin{figure}[h!]
    \centering
    \includegraphics[width=\textwidth]{./images/movisens/Bending1.png}
    \caption{EDA Trace for the first Bending Fall}
    \label{fig:movisens-bending1}
\end{figure}

\begin{figure}[h!]
    \centering
    \includegraphics[width=\textwidth]{./images/movisens/Bending2.png}
    \caption{EDA Trace for the second Bending Fall}
    \label{fig:movisens-bending2}
\end{figure}

\vspace{15mm}
Even for the bending falls, the results obtained were poorly significant. The first fall, which is reported in figure \ref{fig:movisens-bending1}, shows a very limited increase of the Skin Conductance Level while the Skin Conductance Response describes an irregular trace that does not provide meaningful event-related information. The same observations apply even for the second fall, which is reported in figure \ref{fig:movisens-bending2}. Despite the fact that the Skin Conductance Level depicts a consistent increase of 2 $\mu S$, the Skin Conductance Response does not provide information that allow to locate a specific stimulation. A similar trace can be achieved while performing physical activity without any external stimulation that increases the levels of stress and fear of an individual.

\newpage

\subsubsection{Sidewards Fall}\label{subsubsec:sidewards-fall}

The \textbf{Sidewards Fall} was performed as one of the most common falls that involve every kind of individuals, from young ones to elderlies. The activity involves falling respectively on the right and left side from a standing position.
%The obtained results are reported below, respectively in figure \ref{fig:movisens-sidewardsdx} and \ref{fig:movisens-sidewardssx}.

\begin{figure}[h!]
    \centering
    \includegraphics[width=\textwidth]{./images/movisens/SidewardsDX.png}
    \caption{EDA Trace for the SidewardsDX Fall}
    \label{fig:movisens-sidewardsdx}
\end{figure}

\begin{figure}[h!]
    \centering
    \includegraphics[width=\textwidth]{./images/movisens/SidewardsSX.png}
    \caption{EDA Trace for the SidewardsSX Fall}
    \label{fig:movisens-sidewardssx}
\end{figure}

In both cases, the EDA trace results irregular and does not describe relevant activity patterns for classification purposes. The right fall, reported in figure \ref{fig:movisens-sidewardsdx}, does not show increases in the tonic component, while the Skin Conductance Response results irregular and lacks of clear peak patterns. The left fall, which is instead reported in figure \ref{fig:movisens-sidewardssx}, does not describe a meaningful increase of the tonic component, while the phasic one does not show any relevant peak and, besides, includes a visible gap which may have been caused by a momentarily detachment of one of the two electrodes. Even though the latter where properly attached to the epidermis, the issues regarding possible detachments constitute a substantial problematic that is difficult to address in high-motion activity contexts.

\subsubsection{Backwards Fall}\label{subsubsec:backwards-fall}

Similarly to the first experimentation session, two backwards falls were performed. These were both based on the activity of falling posteriorly, which normally involves people of variable age and physical conditions.
% The results obtained are respectively reported in figures \ref{fig:movisens-backwards1} and \ref{fig:movisens-backwards2}.

\begin{figure}[h!]
    \centering
    \includegraphics[width=\textwidth]{./images/movisens/Backwards1.png}
    \caption{EDA Trace for the first Backwards Fall}
    \label{fig:movisens-backwards1}
\end{figure}

\begin{figure}[h!]
    \centering
    \includegraphics[width=\textwidth]{./images/movisens/Backwards2.png}
    \caption{EDA Trace for the second Backwards Fall}
    \label{fig:movisens-backwards2}
\end{figure}

In both the observed cases, reported respectively in figure \ref{fig:movisens-backwards1} and \ref{fig:movisens-backwards2}, the phasic component shows more evident peaks while the tonic one describes a slight increase. Similarly to what have been observed in the first experimentation, the backwards fall depicts a clearer EDA trace in relation to the other ones observed. It is reasonable to assume that the cause may rely on the fact that, as stated previously, the psychophysiological reaction for a fall in which the individual is not able to see the ground may be different. However, it is not possible to discriminate the variations induced by the movements detected by the electrodes and similar behaviours should be further investigated.

\subsubsection{Front Fall}\label{subsubsec:front-fall}

\begin{figure}[h]
    \centering
    \includegraphics[width=\textwidth]{./images/movisens/Front.png}
    \caption{EDA Trace for the Front Fall}
    \label{fig:movisens-front}
\end{figure}

The last activity performed was the front fall, which required the participant to fall forward on his hands while landing on the ground. The result obtained has been reported in figure \ref{fig:movisens-front}.

In this case, a peak in the Skin Conductance Response is clearly visible. Only minimal variations were, instead, detected within the terms of the Skin Conductance Level. Again, this leads to the supposition that the movement may have affected the event-related measurement by causing the observable peak, especially considering the pattern of the fall performed, which required the participant to oppose resistance with his hands (including the dominant one, whose wrist is the one on which the sensor was worn). Even in this case, conclusions cannot be drawn without further investigations.

\subsubsection{Considerations}\label{subsubsec:further-observations}

In order to provide a complete analysis of the results obtained, the complete measurement of the whole experimentation sessions was also examined. For this purpose, only the \textbf{filtered Raw signal} and the subsequently extracted \textbf{Tonic Component} were observed.

More specifically, the maximum difference in the EDA values detected during the whole session corresponds to \textbf{13.9103} $\mu S$. The diagram provided in figure \ref{fig:movisens-global} proves, through the computation of the Tonic Component, that this variation happened slowly with an almost constant increasing pattern during the experimentation. This observation corresponds to the original expectations. More specifically, the participant performed several falls in a short period of time and, according to the available literature, \textbf{the Electrodermal Activity is supposed to increase during physical activity} \cite{eda-interval-4}.

\vspace{15mm}

\begin{figure}[h]
    \centering
    \includegraphics[width=\textwidth]{./images/movisens/Global.png}
    \caption{EDA Trace for the whole Measurement Session}
    \label{fig:movisens-global}
\end{figure}
