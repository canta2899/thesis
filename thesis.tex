%% Le lingue utilizzate, che verranno passate come opzioni al pacchetto babel. Come sempre, l'ultima indicata sar� quella primaria.
\def\thudbabelopt{italian, english}
\documentclass[target=bach, hidelinks]{thud}
\setcounter{tocdepth}{3}
\setcounter{secnumdepth}{3}

\course{Internet of Things, Big Data e Web}
\title{Feasibility of Electrodermal Activity Data Acquisition for Fall Detection Systems using Wearable Devices}
\author{Andrea Cantarutti}
\supervisor{Prof.\ Vincenzo Della Mea}
% \cosupervisor{Arch.\ Rambaldo Melandri \and Dott.\ Giorgio Perozzi}
%% Altri campi disponibili: \reviewer, \tutor, \chair, \date (anno accademico, calcolato in automatico), \rights (\and per nomi separati)

%% --- Pacchetti consigliati ---
%% pdfx: per generare il PDF/A per l'archiviazione. Necessario solo per la versione finale
\usepackage[a-1b]{pdfx}
\usepackage[pdfa]{hyperref}

\usepackage{array}
\usepackage{geometry}
\usepackage{minted}
\usepackage{amsmath}
\usemintedstyle{colorful}
\usepackage{eurosym}
%% tocbibind: Inserisce nell'indice anche la lista delle figure, la bibliografia, ecc.

%% --- Stili di pagina disponibili (comando \pagestyle) ---
%% sfbig (predefinito): Apertura delle parti e dei capitoli col numero grande; titoli delle parti e dei capitoli e intestazioni di pagina in sans serif.
%% big: Come "sfbig", solo serif.
%% plain: Apertura delle parti e dei capitoli tradizionali di LaTeX; intestazioni di pagina come "big".

\begin{document}
\maketitle

\pagestyle{big}

%% Dedica (opzionale)
\begin{dedication}
    To be written 
\end{dedication}

\acknowledgements
To be written
%% Sommario (opzionale)
% \abstract

%% Indice
\tableofcontents

%% Lista delle tabelle (se presenti)
\listoftables

%% Lista delle figure (se presenti)
\listoffigures

%% Corpo principale del documento
\mainmatter

%% Parte
%% La suddivisione in parti � opzionale; solitamente sono sufficienti i capitoli.
%\part{Parte}

\chapter{Introduction}
\label{ch:introduction}

Falling may be experienced as one of the most damaging events, especially for elderly people. According to the World Health Organization Newsroom \cite{WhoData}, an estimated 684.000 individuals die every year from falls globally, while 37.3 million falls turn out to be severe enough to require medical attention. 
Over the years, an urgent need for fall detection system has led researchers to the development of increasingly accurate solutions involving the usage of different sensors and various techniques for data processing, collection and analysis.
Furthermore, thanks to the rapid development of the Internet of Things, the employment of \textbf{sensors networks} allowing the fusion of heterogeneous data has been regarded as an effective method to address the problem of fall detection.

Whereas data retrieved from devices such as \textbf{Inertial Measurement Units} constitutes the primary area of study, over the past few years researchers have developed a growing interest towards the analysis of bio-signals in order to depict the consequences that a fall may cause to the human body both at the \textit{physical} and \textit{mental} level.

This dissertation describes the most advanced and accurate results obtained by the research community and analyzes the feasability of the analysis of the \textbf{Electrodermal Activity} to detect the variation of stress condition during a fall. 

% TODO cite https://www.who.int/news-room/fact-sheets/detail/falls
% TODO furthermore or additionally?
% TODO describe chapters


\chapter{Background}
\label{ch:background}

In order to define a common ground for the reader, this chapter introduces the main contents on which this work is based. For this purpose, a brief introduction about the current \textbf{state of knowledge} in terms of fall detection and the role of the \textbf{Electrodermal Activity} is proposed. Additionally, the concept of \textbf{Wearable Device} is briefly introduced.

% TODO evaluate the title, might need to be changed
\section{Benefits and Progress of Fall Detection Systems }\label{sec:sectionname}

% file:///Users/andrea/Downloads/sensors-21-05134-v2.pdf
In the field of research, a fall is described as an \textbf{unpredicted event} leading a subject to rest on the floor level~\cite{Lamb1}. On that basis, the activity of Fall Detection has referred over the years to the identification of a fall through cameras and sensors in order to request assistance.

Furthermore, various classifications of falls have been outlined. An accurate set of categories was described by Chan \textit{et al., 2018}~\cite{Chen1}, whose study about the specifications of built-in accelerometers in smartphones on fall detection performance divides falls in the following categories: 

\begin{itemize}
    \item Fall lateral left and lie on the floor
    \item Fall lateral left and sit up from the floor
    \item Fall lateral right and lie on the floor
    \item Fall lateral right and sit up from the floor
    \item Fall forward and lie on the floor
    \item Fall backward and lie on the floor
\end{itemize}

Researchers had also pointed out other important aspects that should be considered while evaluating falls, such as the duration of the fall and the age, gender, mental condition and physical condition of the individual.




% https://neurokit2.readthedocs.io/en/latest/examples/eda.html
\chapter{Analysis}
\label{ch:analysis}

With the purpose of providing a deeper understanding on how some of the most relevant datasets regarding fall detection were collected, an analysis was performed in order to identify common \textbf{patterns}, the \textbf{technologies} employed and the \textbf{results obtained}. Afterwards, the previously mentioned \textbf{Electrodermal Activity} and its potential implications in the context of fall detection were both investigated.

\section{Typical biometric data employed in fall detection}\label{sec:hardware}

The consequences of falling may be observed through a series of data that addresses aspects related to physical, physiological and environmental variables. Some of the repercussions that a hard fall may cause are:

\begin{itemize}
    \item A fast shift of the gravitational acceleration values
    \item A change of the altitude above ground level
    \item Feelings of fear and obfuscation (especially for elderly people in severe cases)
    \item A decrease of the body temperature (in cases in which the individual remains prone on the ground for extended periods of time)
\end{itemize}

The following section introduces some of the most commonly used instruments to retrieve data that may be functional to fall detection systems.

\subsection{Accelerometer}\label{subsec:accelerometer}

The accelerometer provides a measure of the \textbf{acceleration} in relation to an entity in its coordinate system. Once calibrated, the obtained value measures 9.81 $m/s^2$ at rest (which corresponds to the gravity acceleration) and drops at 0 $m/s^2$ during a \textbf{free fall}.

Furthermore, modern accelerometer are commonly implemented as \emph{micro-electro-mechanical-systems} and their dimensions have packaged sizes of only 2 x 2 x 1 \textit{mm}. This makes them particularly suitable for \textbf{wearable devices} and \textbf{embedded systems}.

Since a 3-axis accelerometer provides a separate trace for the $x$, $y$ and $z$ axis, a \textbf{magnitude vector} can be computed in order to represent the measurement as a scalar value. The formula involves the calculation of the norm of the coordinate vector and is generally computed in real-time on embedded systems in order to provide a value for classification purposes.

\newcommand\norm[1]{\left\lVert#1\right\rVert}

\begin{figure}[h]
    \begin{equation}
    \norm{a}_2 = \sqrt{a_{x}^2 + a_{y}^2 + a_{z}^2}
    \end{equation}
    \caption{Magnitude Vector formula}
    \label{fig:magnitude}
\end{figure}

\subsection{9-Axis IMUs}\label{subsec:imus}

The \textbf{inertial motion sensor} units (commonly referred as IMUs) provide a combination of three sensors:

\begin{itemize}
    \item A 3-axis accelerometer
    \item A 3-axis gyroscope 
    \item A 3-axis magnetometer 
\end{itemize}

While the accelerometer and gyroscope signals provide measures to describe the \emph{rotation} and the \emph{acceleration} around each axis, a magnetometer is employed in order to sense the surrounding \emph{magnetic field} and correct small drifts over long lasting periods of time. 

The combination of the latter sources provides a criteria to compute the \textbf{complete orientation in space} and offers remarkable advantages in order to improve accuracy in motion tracking and fall detection systems.

\subsection{Barometric Altimeter}\label{subsec:altimeter}

The barometric altimeter determines changes in elevation by employing a pressure sensor. Compared to the changes in the atmospheric pressure, in fact, the altitude variation result inversely proportional \cite{mems-altimeter}.

Although barometric altimeters are involved in a multitude of usages the data collected may provide useful information in the context of fall detection systems. The altitude level, when combined with the data retrieved from a 9-Axis IMU or an accelerometer, may confirm that a fall event just happened with higher levels of accuracy.

Several instruments for motion tracking include both a 9-Axis inertial measurement unit and a barometric altimeter. For that, they are commonly referred as \textbf{10-Axis IMUs}.

\subsection{Biometric Sensors}\label{subsec:biometric-sensors}

Another branch of information which has been widely regarded lately is related to biometric sensors. These include a variety of instruments to collect biometric signals by using appropriate hardware, such as \textbf{electrodes}, \textbf{skin contact technologies} and others.

Besides some units require the usage of specific hardware, other sensors (such as the ECG, EEG, Temperature, EDA and others) have already been implemented in several commercial wearable devices and provide accurate data that can later be involved in the computation of several biometric descriptors. 

In the context of fall detection systems, a synchronized retrieval of both biometric and motion related data may significantly improve the accuracy of classification models.

\section{Multimodal datasets for fall detection systems}\label{sec:datasets}

Despite the fact that various falls datasets have been made available throughout the years, two of them were selected for the purpose of this analysis as a consequence of their relevance in the research environment.

\subsection{UMAFall - A Multisensor Dataset for the Research on Automatic Fall Detection}\label{subsec:umafall}

The \textbf{UMAFall} dataset gained considerable interest since its publication (happened in 2017). The primary difference from other datasets was, in fact, related to the way Casilari \textit{et al., 2018}~\cite{umafall} approached the data collection stage, which involved the usage of multiple units of the same sensor. 

Basing on the conclusions drawn by previous publications, UMAFall was designed in order to provide a public dataset to study the importance of sensor units placement for the effectiveness of fall detection algorithms \cite{umafall}. The traces collected provide measurements of the mobility during daily life activities and falls, obtained by \textbf{five sensing nodes} placed on different positions of the body of several individuals.

\subsubsection{Technologies Involved}\label{subsubsec:umafall-technologies}

The data gathering architecture was implemented as a \textbf{Bluetooth Low Energy} (BLE) piconet composed of:

\begin{itemize}
    \item Four wearable sensors located in four different positions of the body, acting as \textbf{slave nodes}
    \item An Android smartphone, acting as the \textbf{master node}
\end{itemize}

The nodes were implemented through multiple \textbf{SimpleLink SensorTag} units. These consist of IoT devices powered by a CC2650 ARM microcontroller that integrates: 

\begin{itemize}
    \item a 2.4 GHz transceiver
    \item 10 embedded sensors, including an MPU-9250 multichip module
\end{itemize}

The latter made possible the retrieval of motion related data, combining the values registered by a 3-axis gyroscope, a 3-axis accelerometer and a 3-axis magnetometer, which were regularly sent to the master unit and later saved in a CSV file. However, the usage of the Bluetooth protocol has demanded low resolutions in order to avoid saturating the communication channel. Therefore, the \textbf{sample rate} was set to 20 Hz for each unit.

\subsubsection{Activities Performed}\label{subsubsec:umafall-activities}

The four sensors were placed on locations typically reported in literature, such as the ankle, waist, chest and right wrist. Furthermore, the participants consisted of seventeen individuals divided in ten males and seven females aged between 18 and 55 years old.

Because of the practical sensor architecture based on wearable devices, data could be retrieved in a domestic environment and included the activities reported in Table \ref{toc:umafall}

\begin{table}[H]
\centering
\begin{tabular}{ll}
    \hline
    Activity                & Category \\
    \hline
    Body bending            & Daily Activities \\
    Climbing stairs down    & Daily Activities \\
    Hopping                 & Daily Activities \\
    Light jogging           & Daily Activities \\
    Lying down              & Daily Activities \\
    Sitting down            & Daily Activities \\
    Walking                 & Daily Activities \\
    Forward fall            & Fall \\
    Later fall              & Fall \\
    Backwards fall          & Fall \\
    \hline
\end{tabular}
\caption{Activities evaluated in UMAFall}
\label{toc:umafall}
\end{table}

\subsubsection{Results Obtained}\label{subsubsec:umafall-results}

Casilari \textit{et al., 2018}~\cite{umafall} provided a dataset including 531 CSV files of which 322 were reporting daily activities data and 209 were reporting falls related data, each one of 15-seconds duration. An initial analysis was performed in order to describe the variation of the \textbf{Signal Magnitude Vector} for each dataset.

Lastly, the results obtained determined substantial difficulties in distinguishing falls from moderate activities using threshold based techniques. An approach based on the fusion of multiple sensor data and the usage of Machine Learning advances in order to reduce the number of \textbf{false positives} obtained was, lastly, proposed by the authors, but not implemented.

\subsection{UP-Fall Detection Dataset: A Multimodal Approach}\label{sec:upfall}

The \textbf{UP-Fall} dataset was presented in 2019 in order to collect fall-related information according to three major modalities:

\begin{itemize}
    \item \textbf{Wearable sensors}
    \item \textbf{Ambient sensors}
    \item \textbf{Vision devices}
\end{itemize}

The aim of the study was providing a considerable amount of data collected from heterogeneous sources in order to address the lack of publicly available measurements for the evaluation of fall detection systems \cite{upfall}.

\subsubsection{Technologies Involved}\label{subsubsec:upfall-technologies}

The hardware involved consisted of: 

\begin{itemize}
\item Five \textbf{Mbientlab MetaSensor} wearables located in different points of the body and collecting data from: 
    \begin{itemize}
        \item A 3-axis accelerometer
        \item A 3-axis gyroscope
        \item An ambient light sensor
    \end{itemize}
\item A \textbf{NeuroSky MindWave} electroencephalograph headset measuring the brainwave signal
\item Six \textbf{infrared sensors} forming a grid above the floor of the room
\item Two \textbf{Microsoft LifeCam Cinema} cameras providing a frontal and a lateral view of the individual 
\end{itemize}

The data gathering architecture was implemented through the usage of two computers and three Raspberry PI V3 in order to collect the information from all the sensors and later save it in the form of multiple CSV files.

In this case a sample rate of \~ 18.4 Hz was configured in order to accommodate the requirements of all the units involved.

\subsubsection{Activities Performed}\label{subsubsec:upfall-activities}

The UP-Fall dataset was collected in a \textbf{controlled environment} where 17 young and healthy subjects were required to perform 11 different activities with three attempts each \cite{upfall}.

\begin{table}[H]
\centering
\begin{tabular}{ll}
    \hline
    Activity                          &   Category           \\
    \hline
    Walking                           &   Daily Activities   \\
    Standing                          &   Daily Activities   \\
    Sitting                           &   Daily Activities   \\
    Picking up an Object              &   Daily Activities   \\
    Laying                            &   Daily Activities   \\
    Jumping                           &   Daily Activities   \\
    Falling sitting in empty chair    &   Fall               \\
    Falling sideward                  &   Fall               \\
    Falling backwards                 &   Fall               \\
    Falling forwards using knees      &   Fall               \\
    Falling forwards using hands      &   Fall               \\
    \hline
\end{tabular}
\caption{Activities evaluated in UP-Fall}
\label{toc:umafall2}
\end{table}

The raw gathered data was later divided in different time windows and, for each one of them, a \textbf{feature extraction and selection} process was performed. The processed information was, then, used to evaluate the performance of four classification models: 

\begin{itemize}
    \item Random Forest
    \item Support Vector Machines
    \item Multi-Layer Perceptron
    \item \textit{k}-Nearest Neighbors
\end{itemize}

The performances of the latter were evaluated through the metrics of \textit{accuracy}, \textit{precision}, \textit{sensitivity}, \textit{specificity} and $F_1$ - \textit{score}.

A limitation pointed out by the authors \cite{umafall} involves the context of the experimentation: the falls performed were self-initiated and different from real-life falls. These kind of aspects constitute a primary concern for researchers because of the difficulties in addressing them and the inaccuracy they might lead to, as stated in \ref{sec:fallintro}.

\subsubsection{Results Obtained}\label{subsubsec:upfall-results}

The dataset obtained and the following activities of processing and subsequent analysis performed led to observe that the data retrieved from the inertial measurement units of the wearables played a major role in the accuracy of the classification models. The accuracy (depicted by the $F_1$-score) of the IMUs-only based classification reached a value of 70.31\% while classifiers trained with combination of infrared and camera data demonstrated considerably lower performances (between 15\% and 33\%). Lastly, the combination of data collected by wearables, cameras and the EEG sensor obtained the highest $F_1$-score accuracy measure, which corresponded to 70.44\%. 

Additionally, a Convolutional Neural Network was trained in order to improve the classification performance based on video recording data. This reached an $F_1$-\textit{score} of 71.2\%.

In conclusion, in the context of fall detection systems, the analysis performed on the UP-Fall dataset demonstrated that a certain degree of accuracy may be reached by processing data from sources of different nature, even though classifications can be improved by approaching the subject in a multimodal and heterogeneous manner.

\section{Electrodermal Activity}\label{sec:eda-description}

In the whole field of bio-signals, a subject of interest in the recent years has been the so-called \textbf{Electrodermal Activity}, also known as \textbf{Skin Conductance} or \textbf{Galvanic Skin Response}.

The latter describes the continuous variations of the electrical conductivity of the skin (which is also referred as Skin Conductance or SC) and has been depicted as the main criteria to investigate the psychophysiological states of an individual since the beginning of the 20th century.

\subsection{EDA signals and correlation with psychophysiological stress detection}\label{subsec:eda-signals}

As stated in section \ref{sec:edaintro}, the Electrodermal Activity provides a non-invasive technique to analyze the activity of the \textbf{autonomic nervous system} (ANS) by measuring the resistance opposed by the skin to the electrical current, which varies according to the \textbf{sweat glands} activity.

Perspiration is, in fact, governed by the sympathetic nervous system \cite{bartholomew} through the postganglionic sudomotor fibers, in connection with the external stimulations received. On these principles, a situation of emotional distress that induces an excitement of the ANS activity would also cause changes in sweat secretion. The measuring of the latter may, then, provide a qualitative indicator related to the \textbf{emotional response} of a subject \cite{carlson}.

\begin{figure}[h]
    \centering
    \includegraphics[width=\textwidth]{./images/eda-cause-effect.drawio.png}
    \caption{Correlation between Electrodermal Activity and the Autonomic Nervous System}
    \label{fig:eda-ans}
\end{figure}

\subsection{Measurement}\label{subsec:eda-measurement}

% https://www.birmingham.ac.uk/Documents/college-les/psych/saal/guide-electrodermal-activity.pdf

In order to detect Electrodermal Activity on and individual's skin, two different approaches may be employed:

\begin{itemize}
    \item Exosomatic Methodology
    \item Endosomatic Methodology
\end{itemize}

The first one tends to be the most commonly used and implies the application of a direct or alternating current directly on the skin. The second one, instead, does not involve the usage of any external current.

% https://www.tobiipro.com/learn-and-support/learn/GSR-essentials/how-does-a-gsr-sensor-work/
The implementation of the exosomatic methodology requires the usage of \textbf{two electrodes} through which a voltage of direct current of 0.5 VDC is trasmitted. After being applied to the skin, the current flow through the electrodes is measured by applying the \textbf{Ohm's Law}, which determines the resistance opposed by the epidermis.

\begin{figure}[h]
    \begin{equation}
    \begin{aligned}
    I = \frac{U}{R} \\
    \\
    R = \frac{U}{I}
    \end{aligned}
    \end{equation}
    \caption{Ohm's Law Formula}
    \label{fig:ohmlaw}
\end{figure}

It is important to place both the electrodes on the palm of the hand, on two adjacent fingers or, otherwise, on the sole of the foot. These areas have been, in fact, identified as the most prominent in terms of perspiration. Modern EDA sensors employ electrodes with Ag/AgCl contact points in order to accurately transmit the electrical current \cite{eda-imotions}. Isotonic gel is also generally used to accommodate the signal transmission from the skin and consequently improve its quality.

\begin{figure}[h]
    \centering
    \includegraphics[width=8cm]{./images/skin-resistance.drawio.png}
    \caption{Representation of the EDA measurement principle}
    \label{fig:eda-ans}
\end{figure}

The resistance obtained is expressed in terms of the \textbf{Siemens} ($S$) measuring unit. More specifically, because of the small values obtained, microSiemens ($\mu S$) are normally employed.

\subsection{Sample rates}\label{subsec:eda-signal-propertie}

EDA is regarded as a slow measure \cite{eda-guide}. According to the related literature, typical frequency intervals are, in fact:

\begin{itemize}
    \item 0 - 3 Hz \cite{biosignalplux-guide}
    \item 0 - 10 Hz \cite{eda-hci}
    \item 0.0167 - 0.25 Hz \cite{eda-interval-3}
    \item 0.045 - 0.25 Hz \cite{eda-interval-4}
    \item 0.05 - 35 Hz \cite{eda-guide}
\end{itemize}

By applying the \textbf{Nyquist Theorem} \cite{nyquist}, which states that a signal should be sampled at least at two times its highest frequency, it can be inferred that EDA signals must be sampled at 70 Hz or more. In practice, sample rates between 70 Hz and 400 Hz are commonly used according to the requirements of the experimentation in order to provide greater accuracy during the signal processing stage \cite{eda-guide}. Lowpass and bandpass filters are, then, applied in order to smooth the signal and remove unwanted \textbf{noises}.

\subsection{Phasic and Tonic component}\label{subsec:phasic-tonic}

EDA signals are characterized by two \textbf{additive} components \cite{eda-guide} referred as:

\begin{itemize}
    \item \textbf{Tonic component} (also known as Skin Conductance Level or SCL)
    \item \textbf{Phasic component} (also known as Skin Conductance Response or SCR)
\end{itemize}

While the tonic component describes the slowly and continuously changing part of the signal, the phasic component depicts, instead, a fast changing signal whose variations are determined by event-related stimulations.

In order to decompose the measurement obtained in its components, an approach based on \textbf{standard deconvolution} can be employed. Assumed the additivity of the two signals, the skin conductance can be represented as follows:

\begin{equation}
    SC = SC_{tonic} + SC_{phasic}
\end{equation}

Furthermore, both the $SC_{tonic}$ and $SC_{phasic}$ components can be represented by a convolution operation which foresees the multiplication of a so-called \textbf{driver component} recorded by the sensor with an Impulse Response function \cite{edasvm}:

\begin{figure}[h]
\begin{equation}
SC = Driver_{tonic} \cdot IRF + Driver_{phasic} \cdot IRF
\end{equation}
\caption{Convolution process of the tonic and phasic components}
\label{fig:eda-convolution}
\end{figure}

In order to detect variations of an individual's arousal over a specific time interval, the phasic component obtained by the deconvolution process is generally employed for feature extraction purposes.

The example reported in Figure \ref{fig:eda-example} illustrates:

\begin{itemize}
    \item A raw EDA signal with a duration of 15 seconds depicting an increase in the arousal level
    \item The subsequently extrapolated SCR component
    \item The SCL component
\end{itemize}

\begin{figure}[h]
    \centering
    \includegraphics[width=\textwidth]{./images/eda-simulation.png}
    \caption{Decomposition of Raw EDA signal in its two components}
    \label{fig:eda-example}
\end{figure}

The raw signals were generated thorugh NeuroKit2, a Python Toolbox for Neurophysiological Signal Processing that implements several biosignal processing routines, including EDA-related ones \cite{neurokit}.

\subsection{Use Cases}\label{subsec:eda-usecases}

The application of psychophysiological sensors is becoming an increasingly frequent practice in the context of Human Computer Interaction \cite{eda-hci}. More specifically, the Electrodermal Activity may provide a measure to define the level of \textbf{arousal} that characterizes and individual during experimentations and, generally, the usage of specific technologies.

Recent studies involved, for example, the measurement of Heart Rate and Electrodermal Activity in order to define an objective evaluation method for experiences based on \textbf{virtual reality}. Moreover, excellent results have been obtained by Sánchez-Reolid \textit{et al., 2018}~\cite{edasvm}, whose classification model based on \textbf{Deep Support Vector Machines} have achieved excellent results in the identification of stress pattern by employing EDA measurements acquired from a wearable device.

\section{Electrodermal Activity and Fall Detection Systems}\label{sec:eda-fall-detection}

The Electrodermal Activity measurement in the context of fall detection has not been a true object of interest yet. It is, however, reasonable to assume that analyzing the skin conductance response may help to provide additional information in order to help classifiers to reach higher levels of accuracy. 

In 2013, T. Horta \textit{et al., 2018}~\cite{eda-fall-detection} included EDA signals in the development of a system for fall detection and prevention based on biofeedback monitoring solutions. The latter obtained valid results, but the analysis performed by the authors did not aim to provide assessments on the influence that Electrodermal Activity had on the accuracy of the whole system (which include other bio-sensors such as electroencephalography, electrocardiogram, electromyography and blood volume pressure).

\subsection{Correlations between EDA and Wearable Devices}\label{subsec:eda-wearables}

Because of the nature of EDA signals, during the years the main object of interest has been monitoring patients over extended periods of time and in real-life situations in order to identify patterns that are difficult to replicate in artificial environments. This has consequently focused the interest of researchers in the development of wearable devices for the electrodermal activity measurement \cite{poh-wearable}. 

In 2010, Poh \textit{et al., 2018}~\cite{poh-wearable} described one of the first implementations of a compact and cost-effective wearable sensor for unobtrusive and long-term assessment of Electrodermal Activity. This was packed inside a wristband exposing two electrodes. The outcomes obtained were strongly correlated with FDA-approved measurement systems. Furthermore, over the last decade several companies such \textbf{Empatica} (an MIT spin-off) or \textbf{Movisens GmbH} (a global leader in ambulatory assessment solutions) implemented highly performing wearable devices for EDA measurement which have been commonly employed in the field of research.

Additionally, at the end of September 2020 \textbf{Fitbit} released \textbf{Sense}, the first commercial wearable device implementing EDA measurement as one of its key features \cite{fitbit-eda}. This opened the way to new opportunities for developers, that will ghopefully be able to build software in order to measure and analyse the Electrodermal Activity by employing Application Programming Interfaces of commercial and commonly diffused devices, without having to rely on expensive and invasive equipment.
% Useful? 

% https://www.ncbi.nlm.nih.gov/pmc/articles/PMC2892750/#:~:text=Electrodermal%20activity%20(EDA)%20refers%20to,et%20al.%2C%201981).




\chapter{Implementation of a Wearable Device for high resolution EDA logging}
\label{ch:implementation}

Besides the realization of EDA sensors results relatively simple when compared with the circuitry of other devices, its limited diffusion at present implies a series of issues that are required to be addressed. More specifically:

\begin{itemize}
    \item EDA wearables oriented to research purposes tend to be particularly expensive (and most of the time unaffordable for small research projects)
    \item Well-known devices implementing EDA-related features do not provide Application Programming Interfaces for developers and do not provide open-source solutions either
\end{itemize}

The following chapter describes the implementation of a \textbf{simple} and completely \textbf{open-source} wearable device, aimed at providing a low-cost framework for the Electrodermal Activity measurement in experimental contexts.

\section{BITalino Electrodermal Activity Sensor}\label{sec:bitalino}

The sensor unit employed is the purpose-built \textbf{Electrodermal Activity Sensor} by \textbf{BITalino}. This constitutes a single module provided by the company aimed at composing, along with many others, the \textbf{BITalino (r)evolution Board Kit} \cite{bitalino-general}, an all-in-one device implementing a custom firmware for the collection and aggregation of data retrieved from multiple biosensors. Besides all the features implemented by these boards result interesting in research contexts, the main drawbacks are related to the lack of mobility that the device would cause, which is a primary concern within the bounds of fall detection systems.

On that basis, a single EDA module was purchased for a 25,00 € (Tax Excluded) price and a minimal Arduino-based firmware was implemented in order to retrieve measurements from the individual unit without having to rely on third-party solutions.

\subsection{Description and Features}\label{sec:bitalino-features}

The EDA sensor module consists of a breakout boards whose dimensions correspond to 12mm x 27mm \cite{bitalino-general}. A complete list of informations regarding the configuration of the unit is reported in Table \ref{toc:bitalino-features} .

\begin{table}[H]
\centering
\begin{tabular}{ll}
    \hline
    Parameter               & Value \\
    \hline
    Current                 & DC \\
    Range                   & 0-30 $\mu S$ \\
    Consumption             & $\pm 0.1 mA$ \\
    Bandwidth               & 0 - 2.8 Hz \\
    Measurement             & continuous \\
    Input Voltage Range     & 1.8 - 5.5 V \\
    \hline
\end{tabular}
\caption{Characteristics of the BITalino EDA sensor unit}
\label{toc:bitalino-features}
\end{table}

Furthermore, the device is sold in the three following configurations, according to the requirements of customers: 

\begin{itemize}
    \item Self-assemble
    \item Self-assemble with UC-E6 connectors
    \item Assembled
\end{itemize}

The first version consists of the single breakout board and requires manual soldering of both the electrodes and connectors, while the second one provides pre-installed UC-E6 connectors in order to facilitate the connection of the electrodes and the main board. The assembled version provides, instead, a ready-made configuration consisting on: 

\begin{itemize}
    \item Pre-soldered electrodes
    \item Pre-soldered UC-E6 cable
    \item A 3D printed ABS case containing the breakout board
\end{itemize}

% https://bitalino.com/storage/uploads/media/homeguide4-eda.pdf

In order to exploit the convenience of both the ABS case and the pre-soldered electrodes cables, a pre-assembled unit was acquired and later modified in order to implement the connection with a third-party microcontroller. 


% \begin{figure}[h]
%     \centering
%     \includegraphics[width=3cm]{./images/bitalino.drawio.png}
%     \caption{Decomposition of Raw EDA signal in its two components}
%     \label{fig:eda-example}
% \end{figure}







\chapter{Collecting fall detection related EDA data}
\label{ch:collection}

In order to further investigate the results obtained during the first experimentation session, a second testing trial was performed. This allowed the collection of additional data in order to draw conclusions in relation to the \textbf{feasibility} of the Electrodermal Activity analysis for classification purposes in the context of fall detection.

In this case, a commercial and publicly acknowledged device was employed in order to overcome the issues concerning the sensitivity to shocks and impacts of the BITalino device and, in addition, to acquire further bio-signals along with the Electrodermal Activity data.

\section{Choosing a High End Electrodermal Activity Sensor}\label{sec:movisens}

The device employed consists of the \textbf{EdaMove 4} sensor by \textbf{Movisens GmbH}, which is one of the most accurate and popular wearable devices aimed at the Electrodermal Activity measurement together with others such as the previously mentioned \textbf{Empatica E4}. In the following sections, a brief overview of the product and its features in proposed.

\subsection{Movisens GmbH}\label{subsec:movisens-company}

Movisens GmbH is a German company deemed as a global leader in ambulatory assessment solutions \cite{movisens}, which offers several services such as workshops, consults, customized products and software in order to support researchers and provide technologies for the healthcare sector.
The current catalog offers several instruments for data collection, integration, processing and analysis. More specifically, a list of the currently available products is reported in table \ref{toc:movisens-products}.

\begin{table}[H]
\centering
\begin{tabular}{ll}
    \hline
    Name                     &  Description \\
    \hline
    \textbf{Move 4}          & A 10-Axis IMU that integrates a temperature sensor \\
    \textbf{EcgMove 4}       & A Move 4 unit that integrates ECG data \\
    \textbf{EdaMove 4}       & A Move 4 unit that integrates EDA data \\
    \textbf{LightMove 4}     & A Move 4 unit that integrates the acquisition of ambient light measurement \\
    \textbf{SensorTrigger}   & A mobile solution for activity-triggered data logging \\
    \textbf{movisensXS}      & A mobile solution for Experience Sampling purposes \\
    \textbf{DataMerger}      & A desktop-based application for the integration of heterogeneous data \\
    \textbf{DataAnalyzer}    & A desktop-based application to analyze and report the acquired data \\
    \hline
\end{tabular}
\caption{List of the products implemented by Movisens GmbH}
\label{toc:movisens-products}
\end{table}

Additionally, the company provides several solutions that allow researchers and students to utilize the needed instrumentation by requiring a free, limited rent for a specific product. In this case, the company has consented a rental of the \textbf{EdaMove 4} sensor for the required period of time. 

\subsection{The EdaMove 4 Sensor}\label{subsec:edamove4}

Across all the products developed by Movisens, the EdaMove 4 is the one that implements measurement functionalities for the Electrodermal Activity. It consists of a 62,3 x 38,6 x 11,5 mm case with a 26 g weight \cite{edamove4} that provides a Micro-USB connection (in order to connect the unit to the \textit{Sensor Manager} software) and two snap-button connectors in order to attach the positive and negative electrodes. Moreover, a proper wrist band with pre-connected electrodes and specific mount points for the EdaMove 4 sensor allows the device to be worn seamlessly on the wrist or the ankle.


\begin{figure}[htp]
    \centering
    \subfloat[Electrodes Placement]{%
        \includegraphics[width=0.4\textwidth]{./images/edamove1.jpeg}%
    }%
    \hfill%
    \subfloat[The Sensor Unit Employed]{%
        \includegraphics[width=0.4\textwidth]{./images/edamove2.jpeg}%
    }%
    \caption{The EdaMove 4 Sensor}
    \label{fig:edamovesensor}
\end{figure}

Other than acquiring EDA measurement, the unit also offers:

\begin{itemize}
    \item A \textbf{3D Accelerometer}
    \item A \textbf{Rotation Rate} sensor
    \item A \textbf{Pressure} sensor
    \item A \textbf{Temperature} sensor
\end{itemize}

More specifically, the unit implements an exosomatic acquisition technique by applying a standard voltage of 0.5 V DC. The internal Analog to Digital Converter provides a 14 bit resolution and the output rate corresponds to 32 Hz for a 8 Hz bandwidth.

Furthermore, the 3D accelerometer provides a $\pm$ 16 g range with an output rate of 64 Hz, while the rotation rate sensor offers a $\pm$ 2000 dps range, with a 64 Hz output rate and a 70 mdps resolution. The pressure sensor provides, instead, a measurement range between 300 and 1100 hPa, with a 8 Hz output rate and a 0.03 hPa resolution. Finally, the temperature sensor provides an output rate of 1 Hz.

The EdaMove 4 mounts a capacious Lithium battery which provides a constant 3.7 voltage and offers a continuous runtime of almost four days. An internal memory of 4 Gigabytes offers, instead, a maximum recording capacity of 4 weeks, allowing the usage of the sensor in varied contexts, from controlled experimentations to uncontrolled, domestic environments. 

Finally, the EdaMove 4 has been employed in several scientific publications that overall validated its functionalities and accuracy.

\subsection{Sensor Usage}\label{subsec:edamove4-usage}

In order to organise a measurement session and subsequently extract the obtained data, Movisens provides a specific software toolchain, which is cross-compatible between all the Move sensors. 

More specifically, the sensors can be configured through the \textbf{Sensor Manager} software, which allows the definition of personal data in relation to the individual and the specification of parameters such as the \textbf{duration} and the \textbf{start time} of the following measurement. Furthermore, the software detects whether the unit has non-retrieved measurements in its storage and provides functionalities to transfer the acquired dataset to the machine of the user. The data acquisition criteria can be configured in order to match one of the two modalities proposed: 

\begin{itemize}
    \item \textbf{Plain CSV} files
    \item \textbf{Unisens} File Format 
\end{itemize}

Besides Plain CSV is included as an option, the company highly suggests the usage of the Unisens format in order to exploit the functionalities that the standard implements. The latter is, in fact, compatible with both the \textbf{Unisens Viewer} and the \textbf{Data Analyzer} software. The first consists of a free and convenient tool that allows to plot and pre-process the raw data acquired (providing functionalities to remove unwanted sections and set specific markers in order to identify events of interest). The second one, instead, consists of a paid software that implements several algorithms and data processing techniques for the analysis of all the data acquired by Movisens products.

\subsection{The Unisens Data Format}\label{subsec:unisens}

One of the main difficulties that are required to be addressed when collecting data from multiple and heterogeneous sources is related to the storage and representation of the acquired information. More specifically, data can be retrieved at different sample rates or at specific time intervals and requires to be later synchronized e recomposed in order to be represented, analyzed and, eventually, classified. This problematic has been addressed in a multitude of different ways by researches. For example, as stated in \ref{subsubsec:upfall-technologies}, in the case of the UP-Fall dataset, a unique sample rate was defined in order to allow data synchronization from different sources. This strategy, however, required to lower the sample rate of certain devices which could have provided much more accuracy by configuring higher sample rates.

With these aims, the \textbf{FZI Research Center for Information Technology} and the \textbf{Institute for Information Processing Technology} at the University of Karlsruhe developed \textbf{Unisens}, a universal format standard that provides a shared criteria to collect multi sensor data \cite{unisens}. The latter is licensed under the LGPL license and provides the following abilities and features: 

\begin{itemize}
    \item Handling of continuous signals, as well as annotations and values
    \item Separate storage of multiple data channels
    \item Ability to synchronize data collected at various sample rates
    \item A layer of separation between raw data and meta information
    \item A flexible criteria to organise data in logical groups
    \item A human and machine readable meta information source file
    \item Ease of use, even in the case of embedded systems
\end{itemize}

The structure of a Unisens dataset involves the presence of several independent data sources stored as CSV or binary files which are, then, \textbf{aggregated} through a single XML file. The latter provides generic meta information in relation to additional sensor-related data, the initial time-stamp of the measurement, the source file of each data channel (with the required parameters in order to allow a correct parsing) and more.

\vspace{5mm}

\begin{figure}[h!]
    \centering
    \includegraphics[width=15cm]{./images/unisens_viewer.png}
    \caption{The Unisens Viewer Application}
    \label{fig:unisens-viewer-gui}
\end{figure}

As stated previously, the Movisens software toolchain provides a complete integration of the Unisens data format, which was therefore used in order to acquire and store fall detection related data. Additionally, Unisens provides support for common general-purpose programming languages such as \textbf{Python}, \textbf{Java}, \textbf{C\#} and \textbf{Matlab} other than offering the previously mentioned Unisens Viewer software, a desktop based application compatible with the Windows Operative System that allows users to interact with datasets through a graphical user interface. An extract of the content of a Unisens XML file has been reported in figure \ref{fig:unisens-xml}.

\vspace{20mm}

\begin{figure}[H]
\begin{minted}{xml}

<?xml version="2.0" encoding="UTF-8" standalone="no"?>
<unisens comment="Description of the measurement" duration="1186.0"
         timestampStart="2021-12-02T15:12:01.569"
         ...>
    <customAttributes>
        <customAttribute key="sensorLocation" value="right_wrist"/>
        ...
        <customAttribute key="sensorTimeDrift" value="-3.7783"/>
    </customAttributes>
    <signalEntry adcResolution="16"
                 comment="acc"
                 contentClass="acc"
                 dataType="int16"
                 id="acc.bin"
                 lsbValue="0.00048828125"
                 sampleRate="64"
                 unit="g">
        <binFileFormat endianess="LITTLE"/>
        <channel name="accX"/>
        <channel name="accY"/>
        <channel name="accZ"/>
    </signalEntry>
    ...
</unisens>
\end{minted}
\caption{A brief extract of a Unisens XML file}
\label{fig:unisens-xml}
\end{figure}

\newpage

\section{Data Collection}\label{sec:edamove4-data-collection}

\subsection{Organisation of the Session}\label{subsec:session-org}

The data collection session was performed at the \textbf{Simulation Center} of the Santa Maria della Misericoria Hospital in Udine and involved the participation of a single 49 years old male and healthy individual. A single EDA Move 4 sensor was employed and worn on the wrist of the dominant arm of the participant, as depicted in figure \ref{fig:edamovesensor}. All the activities performed have been reported in table \ref{toc:falls-performed-edamove}. Moreover, like for the previous session, it is important to note that all the falls were self-initiated and performed on purpose in a completely safe environment. Finally, in some cases the same fall was performed multiple times.

\begin{table}[H]
\centering
\begin{tabular}{ll}
    \hline
    Name         & Description \\
    \hline
    Bed          & Falling while getting up from a bed \\
    BedSX        & Falling off the left side of the bed while getting up\\
    BedDX        & Falling off the right side of the bed while getting up \\
    SlumpingSX   & Slumping on the left side of the bed while falling  \\
    SlumpingDX   & Slumping on the right side of the bed while falling  \\
    SidewardsSX  & Standing and subsequently falling on the left side \\ 
    SidewardsDX  & Standing and subsequently falling on the right side \\ 
    Bending      & Falling while picking up an object from the ground \\ 
    Backwards    & Standing and subsequently falling backwards \\
    Front        & Standing and subsequently falling anteriorly  \\
    \hline
\end{tabular}
\caption{List of the falls performed}
\label{toc:falls-performed-edamove}
\end{table}

\subsection{Techniques Adopted}\label{subsec:session-techniques}

In this case, a \textbf{unique} and \textbf{continuous} measurement was acquired for the whole duration of the session. This strategy was adopted in order to provide the ability to analyze single event-related episodes as well as the whole evolution of the Electrodermal Activity throughout the entire duration of the experimentation. Moreover, timestamps were acquired in order to mark the beginning and the end of each fall and facilitate the subsequent process of separation of different events.

The acquired data (which also included information retrieved from the other sensors that compose the EdaMove 4) was, then, extracted and preprocessed in order to extract the raw EDA data (which was already converted in $\mu S$ by the sensor itself) and separate each fall. With these aims, a data processing algorithm was programmed and executed on a \textbf{Python Jupyter Notebook} by using the \textbf{PyUnisens} library.

Since timestamps have been acquired (by employing a stopwatch) in order to mark the beginning and the ending of each measurement, these were used in order to extract the data related to each fall. Since the EDA trace obtained from the Unisens dataset, as a measurement of fixed sample rate, did not provide time-related information, the \textbf{Measurement Start Timestamp} was used. This constitutes part of the standard meta information included in the Unisens XML file. More specifically, the EDA signal (which, in Python, is decoded as a Numpy array) was grouped in sub-arrays of fixed size, corresponding to the sample rate (which is, as stated in \ref{subsec:edamove4}, of 32 Hz). A time indication was, then, assigned (with an increment rate of one second) to every group. The latter procedure made possible the usage of the acquired time information in order to isolate each measurement.

\subsection{Results}\label{subsec:results}

The following sections report the most relevant results observed, which have been analyzed and plotted using the Python Programming Language and the previously mentioned \textbf{NeuroKit2} library. For this purposes, minor modifications were applied to the NeuroKit2 source code in order to fine tune specific parameters that provide cleaner results. All the plots were, then, produced by integrating the functionalities of the previously mentioned notebook. 

\subsubsection{Bed Fall}\label{subsubsec:bed-fall}

The first fall performed was the so-called \textbf{Bed Fall}. Three different variants of the latter were performed, which together reflect the most common bed-related fall that involve the majority of people, especially in the case of elderlies. These respectively concern: 

\begin{itemize}
    \item The act of falling while getting up from the bed, which may be caused by an unexpected inability to stand up 
    \item The act of falling due to the attempt to reach an object on the floor while lying on the bed, respectively on the left and right size
\end{itemize}

The results obtained have been reported in figures \ref{fig:movisens-bed}, \ref{fig:movisens-beddx} and \ref{fig:movisens-bedsx}. More specifically, with the same methodology adopted for Chapter \ref{ch:implementation}, the two signals components were separated through a deconvolution approach and plotted separately, together with the raw signal.

\begin{figure}[h!]
    \centering
    \includegraphics[width=\textwidth]{./images/movisens/Bed.png}
    \caption{EDA Trace for the Bed Fall}
    \label{fig:movisens-bed}
\end{figure}

The trace corresponding to the \textbf{Bed Fall} does not depict any increase in the levels of stress or arousal. Besides the raw signal shows a slight increase, it is clear from the separation of the two components that this is completely dictated by the influence of the phasic component. Because the Skin Conductance Level does not show any increase during the whole measurement duration, it is reasonable to assume that the activity described by the phasic component is mainly influenced by the hand movements and the impact of the sensor with the ground.

\begin{figure}[h!]
    \centering
    \includegraphics[width=\textwidth]{./images/movisens/BedDX.png}
    \caption{EDA Trace for the BedDX Fall}
    \label{fig:movisens-beddx}
\end{figure}

\begin{figure}[h!]
    \centering
    \includegraphics[width=\textwidth]{./images/movisens/BedSX.png}
    \caption{EDA Trace for the BedSX Fall}
    \label{fig:movisens-bedsx}
\end{figure}

The traces for the two other falls (respectively, the \textbf{BedDX} and \textbf{BedSX} one) have, instead, a similar pattern. More specifically, a \textbf{slight} and therefore \textbf{poorly significant} increase is observed for the tonic component, while a varying and eventful trace is described by the phasic component. Besides the raw signal of both falls shows an increase of almost 2 $\mu S$, it is possible to assume (by observing the tonic activity) that the cause may be mainly found in the hand movements produced by the individual. In both cases, in fact, the participant landed on the ground by actively putting all of his weight on the hands and therefore causing the electrodes to acquire more event-related stimuli.

In all the three cases examined, the Electrodermal Activity measurement did not, therefore, provide a clear and useful pattern for classification purposes.

\subsubsection{Slumping Fall}\label{subsubsec:slumping-fall}

After the bed falls, the slumping falls were performed. These involve the activity of slumping on the ground while failing the attempt of using the bed as a support in order to avoid keeling over. The action configures itself as a pattern that involves elderly and disabled people. Even in this case, the fall was performed for both the left and right side.

\begin{figure}[h!]
    \centering
    \includegraphics[width=\textwidth]{./images/movisens/SlumpingSX.png}
    \caption{EDA Trace for the SlumpingSX Fall}
    \label{fig:movisens-slumpsx}
\end{figure}

\begin{figure}[h!]
    \centering
    \includegraphics[width=\textwidth]{./images/movisens/SlumpingDX.png}
    \caption{EDA Trace for the SlumpingDX Fall}
    \label{fig:movisens-slumpdx}
\end{figure}

Both the measurements show a slight increase in the Skin Conductance Level, which was higher in the case of the right fall. At the same time, though, the right fall describes a more irregular activity of the phasic component. The latter may have been caused by motion artifacts happened during the falls and consequently have influenced the growth of the signal level. Both the traces do not, however, describe an EDA trace referable to a variation in the emotional condition, which would provide clear and isolated peaks with a significant increase of the SCL component.

\subsubsection{Bending Fall}\label{subsubsec:bending-fall}

The \textbf{Bending Fall} was, then, performed. The latter involves the activity of falling while trying to reach an object in order to grab it. It is, again, a common pattern for elderly people, but, may result significant even for young and healthy individuals.

\vspace{10mm}

\begin{figure}[h!]
    \centering
    \includegraphics[width=\textwidth]{./images/movisens/Bending1.png}
    \caption{EDA Trace for the first Bending Fall}
    \label{fig:movisens-bending1}
\end{figure}

\begin{figure}[h!]
    \centering
    \includegraphics[width=\textwidth]{./images/movisens/Bending2.png}
    \caption{EDA Trace for the second Bending Fall}
    \label{fig:movisens-bending2}
\end{figure}

\vspace{15mm}
Even for the bending falls, the results obtained were poorly significant. The first fall, which is reported in figure \ref{fig:movisens-bending1}, shows a very limited increase of the Skin Conductance Level while the Skin Conductance Response describes an irregular trace that does not provide meaningful event-related information. The same observations apply even for the second fall, which is reported in figure \ref{fig:movisens-bending2}. Despite the fact that the Skin Conductance Level depicts a consistent increase of 2 $\mu S$, the Skin Conductance Response does not provide information that allow to locate a specific stimulation. A similar trace can be achieved while performing physical activity without any external stimulation that increases the levels of stress and fear of an individual.

\newpage

\subsubsection{Sidewards Fall}\label{subsubsec:sidewards-fall}

The \textbf{Sidewards Fall} was performed as one of the most common falls that involve every kind of individuals, from young ones to elderlies. The activity involves falling respectively on the right and left side from a standing position.
%The obtained results are reported below, respectively in figure \ref{fig:movisens-sidewardsdx} and \ref{fig:movisens-sidewardssx}.

\begin{figure}[h!]
    \centering
    \includegraphics[width=\textwidth]{./images/movisens/SidewardsDX.png}
    \caption{EDA Trace for the SidewardsDX Fall}
    \label{fig:movisens-sidewardsdx}
\end{figure}

\begin{figure}[h!]
    \centering
    \includegraphics[width=\textwidth]{./images/movisens/SidewardsSX.png}
    \caption{EDA Trace for the SidewardsSX Fall}
    \label{fig:movisens-sidewardssx}
\end{figure}

In both cases, the EDA trace results irregular and does not describe relevant activity patterns for classification purposes. The right fall, reported in figure \ref{fig:movisens-sidewardsdx}, does not show increases in the tonic component, while the Skin Conductance Response results irregular and lacks of clear peak patterns. The left fall, which is instead reported in figure \ref{fig:movisens-sidewardssx}, does not describe a meaningful increase of the tonic component, while the phasic one does not show any relevant peak and, besides, includes a visible gap which may have been caused by a momentarily detachment of one of the two electrodes. Even though the latter where properly attached to the epidermis, the issues regarding possible detachments constitute a substantial problematic that is difficult to address in high-motion activity contexts.

\subsubsection{Backwards Fall}\label{subsubsec:backwards-fall}

Similarly to the first experimentation session, two backwards falls were performed. These were both based on the activity of falling posteriorly, which normally involves people of variable age and physical conditions.
% The results obtained are respectively reported in figures \ref{fig:movisens-backwards1} and \ref{fig:movisens-backwards2}.

\begin{figure}[h!]
    \centering
    \includegraphics[width=\textwidth]{./images/movisens/Backwards1.png}
    \caption{EDA Trace for the first Backwards Fall}
    \label{fig:movisens-backwards1}
\end{figure}

\begin{figure}[h!]
    \centering
    \includegraphics[width=\textwidth]{./images/movisens/Backwards2.png}
    \caption{EDA Trace for the second Backwards Fall}
    \label{fig:movisens-backwards2}
\end{figure}

In both the observed cases, reported respectively in figure \ref{fig:movisens-backwards1} and \ref{fig:movisens-backwards2}, the phasic component shows more evident peaks while the tonic one describes a slight increase. Similarly to what have been observed in the first experimentation, the backwards fall depicts a clearer EDA trace in relation to the other ones observed. It is reasonable to assume that the cause may rely on the fact that, as stated previously, the psychophysiological reaction for a fall in which the individual is not able to see the ground may be different. However, it is not possible to discriminate the variations induced by the movements detected by the electrodes and similar behaviours should be further investigated.

\subsubsection{Front Fall}\label{subsubsec:front-fall}

\begin{figure}[h]
    \centering
    \includegraphics[width=\textwidth]{./images/movisens/Front.png}
    \caption{EDA Trace for the Front Fall}
    \label{fig:movisens-front}
\end{figure}

The last activity performed was the front fall, which required the participant to fall forward on his hands while landing on the ground. The result obtained has been reported in figure \ref{fig:movisens-front}.

In this case, a peak in the Skin Conductance Response is clearly visible. Only minimal variations were, instead, detected within the terms of the Skin Conductance Level. Again, this leads to the supposition that the movement may have affected the event-related measurement by causing the observable peak, especially considering the pattern of the fall performed, which required the participant to oppose resistance with his hands (including the dominant one, whose wrist is the one on which the sensor was worn). Even in this case, conclusions cannot be drawn without further investigations.

\subsubsection{Considerations}\label{subsubsec:further-observations}

In order to provide a complete analysis of the results obtained, the complete measurement of the whole experimentation sessions was also examined. For this purpose, only the \textbf{filtered Raw signal} and the subsequently extracted \textbf{Tonic Component} were observed.

More specifically, the maximum difference in the EDA values detected during the whole session corresponds to \textbf{13.9103} $\mu S$. The diagram provided in figure \ref{fig:movisens-global} proves, through the computation of the Tonic Component, that this variation happened slowly with an almost constant increasing pattern during the experimentation. This observation corresponds to the original expectations. More specifically, the participant performed several falls in a short period of time and, according to the available literature, \textbf{the Electrodermal Activity is supposed to increase during physical activity} \cite{eda-interval-4}.

\vspace{15mm}

\begin{figure}[h]
    \centering
    \includegraphics[width=\textwidth]{./images/movisens/Global.png}
    \caption{EDA Trace for the whole Measurement Session}
    \label{fig:movisens-global}
\end{figure}


\chapter{Evaluation}
\input{./chapters/evaluation}

\chapter{Conclusions}
\label{ch:conclusions}

This dissertation aimed to provide an assessment related to the feasibility of the Electrodermal Activity data analysis in the context of fall detection.

For this purpose, 



\backmatter % conclusive part

% A summary can be added with \summary

\bibliography{thud}
\bibliographystyle{plain_\languagename} %% Carica l'omonimo file .bst, dove \languagename � la lingua attiva.

\end{document}

