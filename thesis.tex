%% Le lingue utilizzate, che verranno passate come opzioni al pacchetto babel. Come sempre, l'ultima indicata sar� quella primaria.
\def\thudbabelopt{italian, english}
\documentclass[target=bach, hidelinks]{thud}
\setcounter{tocdepth}{3}
\setcounter{secnumdepth}{3}

\course{Internet of Things, Big Data e Web}
\title{Feasibility of Electrodermal Activity Data Acquisition for Fall Detection Systems using Wearable Devices}
\author{Andrea Cantarutti}
\supervisor{Prof.\ Vincenzo Della Mea}
% \cosupervisor{Arch.\ Rambaldo Melandri \and Dott.\ Giorgio Perozzi}
%% Altri campi disponibili: \reviewer, \tutor, \chair, \date (anno accademico, calcolato in automatico), \rights (\and per nomi separati)

%% --- Pacchetti consigliati ---
%% pdfx: per generare il PDF/A per l'archiviazione. Necessario solo per la versione finale
\usepackage[a-1b]{pdfx}
\usepackage[pdfa]{hyperref}

\usepackage{minted}
\usepackage{amsmath}
\usemintedstyle{colorful}
\usepackage{eurosym}
%% tocbibind: Inserisce nell'indice anche la lista delle figure, la bibliografia, ecc.

%% --- Stili di pagina disponibili (comando \pagestyle) ---
%% sfbig (predefinito): Apertura delle parti e dei capitoli col numero grande; titoli delle parti e dei capitoli e intestazioni di pagina in sans serif.
%% big: Come "sfbig", solo serif.
%% plain: Apertura delle parti e dei capitoli tradizionali di LaTeX; intestazioni di pagina come "big".

\begin{document}
\maketitle

\pagestyle{big}

%% Dedica (opzionale)
\begin{dedication}
    To be written 
\end{dedication}

\acknowledgements
To be written
%% Sommario (opzionale)
% \abstract

%% Indice
\tableofcontents

%% Lista delle tabelle (se presenti)
%\listoftables

%% Lista delle figure (se presenti)
\listoffigures

%% Corpo principale del documento
\mainmatter

%% Parte
%% La suddivisione in parti � opzionale; solitamente sono sufficienti i capitoli.
%\part{Parte}

\chapter{Introduction}
\label{section:introduction}

Introduction


\chapter{Background}
\input{./chapters/Background}

% all the other chapters here

\chapter{Conclusions}

To be written

\backmatter % conclusive part

% A summary can be added with \summary

\bibliography{thud}
\bibliographystyle{plain_\languagename} %% Carica l'omonimo file .bst, dove \languagename � la lingua attiva.

\end{document}

